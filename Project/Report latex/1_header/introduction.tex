Video surveillance systems have become increasingly prevalent in modern society, playing a crucial role in enhancing public safety and security. However, the effectiveness of these systems is often hindered by low-resolution footage or poor visual quality, making it challenging to extract and identify individuals of interest accurately. This limitation was highlighted in a recent incident involving a theft at our residence, where the closed-circuit television (CCTV) footage captured the perpetrator's actions but failed to provide a clear and recognizable depiction of their face.

In response to this problem, our project aimed to develop a comprehensive pipeline for extracting and enhancing face images from video footage. The primary objective was to leverage state-of-the-art computer vision and deep learning techniques to improve the visual quality and resolution of face bounding boxes, thereby facilitating better identification and recognition.

Specifically, we employed the YOLOv8 pose detection model to localize human keypoints within video frames, enabling the extraction of face bounding boxes based on detected facial landmarks. Subsequently, we explored various upsampling and super-resolution techniques to enhance the resolution and visual clarity of the extracted face images. This involved experimenting with conventional methods such as Upsampling, Conv2D and Conv2DTranspose layers, as well as leveraging the advanced Image Super-Resolution using Enhanced Super-Resolution Generative Adversarial Networks (ESRGAN) pre-trained model.

While the scope of this project was limited to testing on video footage with relatively higher visual quality than the CCTV recording from the theft incident, our work aimed to demonstrate the potential of deep learning-based techniques for improving face recognition capabilities in surveillance applications. The proposed pipeline contributes to the field of computer vision by offering an automated solution for face extraction and enhancement, potentially addressing the challenges posed by low-resolution or poor-quality video data.

