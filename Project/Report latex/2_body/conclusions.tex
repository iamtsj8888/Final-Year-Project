The project successfully developed a pipeline for face detection, extraction, and enhancement using computer vision techniques. The YOLOv8 object detection model was employed to localize human keypoints within the video frames, enabling the extraction of face bounding boxes based on the detected facial keypoints. The project explored various upsampling and super-resolution techniques, including conventional methods and an advanced approach using the ESRGAN pre-trained model. The experimental results indicated that the ESRGAN model achieved impressive upscaling and enhancement for face images of sufficient resolution but was limited when applied to low-resolution face bounding boxes extracted from videos with poor visual quality.

However, the enhancement techniques employed, while capable of improving the visual quality and resolution of the extracted face bounding boxes, exhibited limitations when dealing with extremely small or degraded face images. The conventional upsampling methods and the ESRGAN model performed optimally when applied to face bounding boxes with sufficient resolution and quality, but their effectiveness diminished as the input quality deteriorated. A key limitation of the current pipeline is its reliance on the availability of a reasonably good quality video feed. While the approach demonstrated promising results on the CCTV footage, it may encounter challenges when dealing with severely degraded or low-resolution videos, where the face bounding boxes become too small or lack sufficient detail for effective enhancement.

However, it is worth noting that the proposed pipeline may still work to some extent with poor quality videos, as the ESRGAN model can still improve the visual quality of face images to a certain extent, even if the results are not as impressive as with higher quality videos. Future work could explore more advanced face extraction methods, specialized super-resolution models tailored for surveillance applications, or the incorporation of additional preprocessing or post-processing steps to address specific challenges encountered in real-world scenarios. 